\chapter{Related Work} \label{related_work}

Details what others have done that is relevant to your work. \ref{intro}.
\begin{enumerate}
    \item Objectives
    \item Describe the context of the research question in detail, defining terminology, and with references.
    \item Explain how the problem, or related problems, has been solved previously. Critically analyze existing solutions. Discuss how your approach compares to these solutions.
    \item Explain other techniques that you have used to: help understand and analyze the research question; motivate your own work; evaluate your solution.
\end{enumerate}
\section{Objectives}
A single sentence that describes the purpose of this section.
\section{Eye Disease Datasets}
\subsection{Feature Datasets}
- Clinical trial dataset'e bak
- From Messidor fundus image, extracted features 
\subsection{Diabetic Retinopathy Fundus Image Datasets}

\citet{kauppi2013constructing} addressed \citet{thacker2008performance}'s 8 key questions for 6 publicly available fundus image databases. These questions are:

\begin{enumerate}
    \item How is testing currently performed?
    \item Is there a data set for which the correct answers are known?
    \item Are there data sets in common use?
    \item Are there experiments which show algorithms are stable and work as expected?
    \item Are there any strawman algorithms?
    \item What code and data are available?
    \item Is there a quantitative methodology for the design of algorithms?
    \item What should we be measuring to quantify performance? What metrics are used?
\end{enumerate}

In his research, he compared these 6 databases and summarised them depends on amount of these addressed questions \citep{kauppi2013constructing}. Also, in his thesis he established DIARETDB1 database by means of these key questions.

In this section, I give information about some of the publicly available fundus image databases.

\subsubsection{DRIVE (Digital Retinal Images for Vessel Extraction)}
DRIVE data set constructed by \citet{staal2004ridge}.
DRIVE data set includes 40 manually labelled retinal images for training and evaluation of \citet{staal2004ridge}'s method which are randomly selected from 400 diabetic subjects between 25-90 years of age as a part of a screening programme in the Netherlands. Images labelled by 3 human observers who were trained by experienced ophthalmologist. 7 of these manually labelled retinal images has DR indications and 33 of them not. 

DRIVE addresses 7 of \citet{kauppi2013constructing}'s key questions which is one of the best results if you compare with the other 6 databases. It makes publicly available DRIVE very popular in automated diabetic retinopaty detection environment.

General information about DRIVE database:
\begin{itemize}
    \item \textbf{Availability date: } 2004
    \item \textbf{Size: } 40 retinal images
    \item \textbf{DR Size: } 7 retinal images
    \item \textbf{Non-DR Size: } 33 retinal images
    \item \textbf{Camera: } Canon CR5 non-mydriatic 3CCD camera with a 45 degree field of view
    \item \textbf{Resolution: } 768 by 584 pixels
\end{itemize}

\subsubsection{STARE (STructured Analysis of the Retina)}
Stare database contains $\sim$400 publicly available raw images which can be obtained from STARE website \citep{STARE}. They also gives a list of information about diagnosis of these images. They also shared 20 hand labelled images for blood vessel segmentation \citep{hoover2000locating} extracted from these $\sim$400 images. Because I can obtain DR, non-DR information from diagnosis list. There are 13 diagnosis which represented with numbers in the list. The images which has DR diagnosis represented with 7 and 8 which addresses Background Diabetic Retinopathy (BDR/NPDR) and Proliferative Diabetic Retinopathy (PDR). These diagnosis and their diagnosis numbers can be found on appendix. {APPENDIX EKLE!!!!} - diagnosis listesini ve image-diagnosis listesini.

General information about STARE database:

\begin{itemize}
    \item \textbf{Availability date: } 2004
    \item \textbf{Size: } 397 retinal images
    \item \textbf{DR Size: } 10 retinal images
    \item \textbf{Non-DR Size: } 10 retinal images
    \item \textbf{Camera: } TopCon TRV-50 fundus camera with a 35 degree field of view
    \item \textbf{Resolution: } 605 by 700 pixels
\end{itemize}

\subsubsection{CHASE (Child Heart And Health Study in England)}

CHASEDB, retinal vessel reference data-set, includes 28 retinal fundus images of 14 multiethnic children's (aged 10 years) both eyes during September 2007 in North-East London as a part of \citep{fraz2012ensemble}. As we mentioned in Chapter \ref{intro}, detecting neovascularisation is another way of diagnosing DR. CHASEDB mostly used for detection of blood vessels \citep{liskowski2016segmenting} \citep{elbalaoui2016automatic}. There is not any information about DR on these images and I am not checking information related with blood vessels so I did not considered this database in my experiments.

General information about CHASE database:
\begin{itemize}
    \item \textbf{Availability date: } 2012
    \item \textbf{Size: } 28 retinal images
    \item \textbf{Camera: } NM-200-D fundus camera with 30 degree field of view made bu Nidek Co. Ltd., Gamagori, Japan
    \item \textbf{Resolution: } 1280 by 960 pixels
\end{itemize}

\subsubsection{ROC}
General information about ROC database:
\begin{itemize}
    \item \textbf{Availability date: } 2004
    \item \textbf{Size: } 40 retinal images
    \item \textbf{DR Size: } 7 retinal images
    \item \textbf{Non-DR Size: } 33 retinal images
    \item \textbf{Camera: } Canon CR5 non-mydriatic 3CCD camera with a 45 degree field of view
    \item \textbf{Resolution: } 768 by 584 pixels
\end{itemize}

\subsubsection{E-Optha}
\subsubsection{HRFDB}

\subsubsection{CMIF}
General information about CMIF database:
\begin{itemize}
    \item \textbf{Availability date: } 2004
    \item \textbf{Size: } 40 retinal images
    \item \textbf{DR Size: } 7 retinal images
    \item \textbf{Non-DR Size: } 33 retinal images
    \item \textbf{Camera: } Canon CR5 non-mydriatic 3CCD camera with a 45 degree field of view
    \item \textbf{Resolution: } 768 by 584 pixels
\end{itemize}

\subsubsection{REVIEW}
General information about REVIEW database:
\begin{itemize}
    \item \textbf{Availability date: } 2004
    \item \textbf{Size: } 40 retinal images
    \item \textbf{DR Size: } 7 retinal images
    \item \textbf{Non-DR Size: } 33 retinal images
    \item \textbf{Camera: } Canon CR5 non-mydriatic 3CCD camera with a 45 degree field of view
    \item \textbf{Resolution: } 768 by 584 pixels
\end{itemize}

\subsubsection{DIARETDB}
General information about DIARETDB0 database:
\begin{itemize}
    \item \textbf{Availability date: } 2004
    \item \textbf{Size: } 40 retinal images
    \item \textbf{DR Size: } 7 retinal images
    \item \textbf{Non-DR Size: } 33 retinal images
    \item \textbf{Camera: } Canon CR5 non-mydriatic 3CCD camera with a 45 degree field of view
    \item \textbf{Resolution: } 768 by 584 pixels
\end{itemize}

General information about DIARETDB1 database:
\begin{itemize}
    \item \textbf{Availability date: } 2004
    \item \textbf{Size: } 40 retinal images
    \item \textbf{DR Size: } 7 retinal images
    \item \textbf{Non-DR Size: } 33 retinal images
    \item \textbf{Camera: } Canon CR5 non-mydriatic 3CCD camera with a 45 degree field of view
    \item \textbf{Resolution: } 768 by 584 pixels
\end{itemize}

General information about DIARETDB2 database:
\begin{itemize}
    \item \textbf{Availability date: } 2004
    \item \textbf{Size: } 40 retinal images
    \item \textbf{DR Size: } 7 retinal images
    \item \textbf{Non-DR Size: } 33 retinal images
    \item \textbf{Camera: } Canon CR5 non-mydriatic 3CCD camera with a 45 degree field of view
    \item \textbf{Resolution: } 768 by 584 pixels
\end{itemize}

\subsubsection{MESSIDOR}
General information about MESSIDOR database:
\begin{itemize}
    \item \textbf{Availability date: } 2004
    \item \textbf{Size: } 40 retinal images
    \item \textbf{DR Size: } 7 retinal images
    \item \textbf{Non-DR Size: } 33 retinal images
    \item \textbf{Camera: } Canon CR5 non-mydriatic 3CCD camera with a 45 degree field of view
    \item \textbf{Resolution: } 768 by 584 pixels
\end{itemize}

\section{Machine Leearning Approaches for Diabetic Retinopathy}
\subsection{Other Approaches}
\subsection{Neural Network Approaches for Diabetic Retinpathy}
